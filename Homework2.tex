
\documentclass{oisinclass}
\usepackage{amssymb}
	\title{MATH143C: Homework 2}
	\author{Oisin Coveney}
	\renewcommand{\arraystretch}{1.5}
\begin{document}

\maketitle

\textbf{Unfinished problems: 3.1 17, 21}

\section*{Exercise Set 3.1}
\subsection*{1}
\subsubsection*{A: \(f(x) = \cos{x}\)}
\begin{align*}
	P_1(x)                   & = f(0)\frac{x-0.3}{-0.3} + f(0.3)\frac{x}{0.3-0}                                                                                         \\
	P_1(0.45)                & = 0.933005                                                                                                                               \\
	\left|Error_{P_1}\right| & = 0.032558                                                                                                                               \\
	P_2(x)                   & = f(0)\frac{(x-0.3)(x-0.9)}{(0-0.3)(0-0.9)} + f(0.3)\frac{(x-0)(x-0.9)}{(0.3-0)(0.3-0.9)} + f(0.91)\frac{(x-0)(x-0.3)}{(0.9-0)(0.9-0.3)} \\
	P_2(0.45)                & = 0.902455                                                                                                                               \\
	\left|Error_{P_2}\right| & = 0.002008
\end{align*}
\subsubsection*{B: \(f(x) = ln(x+1)\)}
\begin{align*}
	P_1(x)                   & = f(0)\frac{x-0.3}{-0.3} + f(0.3)\frac{x}{0.3-0}                                                                                        \\
	P_1(0.45)                & = 0.393546                                                                                                                              \\
	\left|Error_{P_1}\right| & = 0.021983                                                                                                                              \\
	P_2(x)                   & = f(0)\frac{(x-0.3)(x-0.9)}{(0-0.3)(0-0.9)} + f(0.3)\frac{(x-0)(x-0.9)}{(0.3-0)(0.3-0.9)} + f(0.9)\frac{(x-0)(x-0.3)}{(0.9-0)(0.9-0.3)} \\
	P_2(0.45)                & = 0.375392                                                                                                                              \\
	\left|Error_{P_2}\right| & = 0.003828
\end{align*}

\subsection*{3B}
Error for 1A:
\begin{align*}
	R_1(\xi, x)    & = \frac{f^2(\xi)}{2}(x)(x-0.3)         \\
	R_1(0.3, 0.45) & = 0.045558                             \\
	R_2(\xi, x)    & = \frac{f^3(\xi)}{3!}(x)(x-0.3)(x-0.9) \\
	R_2(0.9, 0.45) & = 0.012452
\end{align*}
Error for 1B:
\begin{align*}
	R_1(\xi, x)  & = \frac{f^2(\xi)}{2}(x)(x-0.3)         \\
	R_1(0, 0.45) & = 0.03375                              \\
	R_2(\xi, x)  & = \frac{f^3(\xi)}{3!}(x)(x-0.3)(x-0.9) \\
	R_2(0, 0.45) & = 0.005063
\end{align*}
\newpage
\subsection*{5}
Using \([x_2, x_4]\) for \(P_1\), and \([x_1, x_3, x_4]\) for \(P_2\)
\subsubsection*{B}
\begin{align*}
	P_1(-\frac{1}{3}) & = 0.3505   \\
	P_2(-\frac{1}{3}) & = 0.162944 \\
	P_3(-\frac{1}{3}) & = 0.174519 \\
\end{align*}
\subsubsection*{D}
\begin{align*}
	P_1(0.9) & = 0.443312 \\
	P_2(0.9) & = 0.436628 \\
	P_3(0.9) & = 0.441985 \\
\end{align*}

\subsection*{7}
\subsubsection*{Error for 5B}
\begin{align*}
	n = 1 \rightarrow                                  & \frac{f^2(\xi)}{2!}(x)(x+0.5)          \\
	                                                   & = \frac{6\xi + 8.002}{2}(x)(x+0.5)     \\
	                                                   & = (3\xi + 4.001)(x)(x+0.5)             \\
	\left(\xi = 0, x = -\frac{1}{3}\right) \rightarrow & =
	0.222277                                                                                    \\
	n = 2 \rightarrow                                  & \frac{f^3(\xi)}{3!}(x)(x+0.25)(x+0.75) \\
	                                                   & = \frac{6}{3!}(x)(x+0.25)(x+0.75)      \\
	                                                   & = (x)(x+0.25)(x+0.75)                  \\
	\left(\xi = 0, x = -\frac{1}{3}\right) \rightarrow & = 0.011574                             \\
\end{align*}

\subsubsection*{Error for 5D}
\begin{align*}
	n = 1 \rightarrow                         & \frac{f^2(\xi)}{2!}(x-0.7)(x-1)                                                                                                                                                                                           \\
	                                          & = \frac{\left(x - 1\right) \left(x - 0.7\right) \left(- e^{2 \xi} \sin{\left(e^{\xi} - 2 \right)} + e^{\xi} \cos{\left(e^{\xi} - 2 \right)}\right)}{2}                                                                    \\
	\left(\xi = 1, x = 0.9\right) \rightarrow & = 0.028160                                                                                                                                                                                                                \\
	n = 2 \rightarrow                         & \frac{f^3(\xi)}{3!}(x-0.6)(x-0.8)(x-1)                                                                                                                                                                                    \\
	                                          & = \frac{\left(x - 1\right) \left(x - 0.8\right) \left(x - 0.6\right) \left(- e^{3 \xi} \cos{\left(e^{\xi} - 2 \right)} - 3 e^{2 \xi} \sin{\left(e^{\xi} - 2 \right)} + e^{\xi} \cos{\left(e^{\xi} - 2 \right)}\right)}{6} \\
	\left(\xi = 1, x = 0.9\right) \rightarrow & = 0.013832
\end{align*}
\subsection*{9}
\[y = 4.25\]
\subsection*{10}
\begin{align*}
	f(x)           & = \sqrt{x - x^2}                                                                                                                                                   \\
	P_2(x)         & = y_0\frac{(x - x_1)(x-x_2)}{(x_0-x_2)(x_0 - x_1)} + y_1\frac{(x - x_0)(x-x_2)}{(x_1-x_0)(x_1 - x_2)} + y_2\frac{(x - x_0)(x-x_2)}{(x_2-x_0)(x_0 - x_1)}           \\
	\text{Given: } & x_0 = 0, x_2 = 1, x = 0.5:                                                                                                                                         \\
	f(0.5)         & = \sqrt{0.25} = \pm 0.5                                                                                                                                            \\
	P_2(x)         & = 0\frac{(0.5 - x_1)(0.5-1)}{(0-1)(0 - x_1)} + \left(\sqrt{x_1 - x_1^2}\right)\frac{(0.5 - 0)(0.5-1)}{(x_1-0)(x_1 - 1)} + 0\frac{(0.5 - 0)(0.5-1)}{(1-0)(0 - x_1)} \\
	               & = \left(\sqrt{x_1 - x_1^2}\right)\frac{-0.25}{x_1^2 - x_1} = \frac{-0.25}{\sqrt{x_1 - x_1^2}}                                                                      \\
\end{align*}
Solving for \(f(0.5) - P_2(0.5) = -0.25\):
\begin{align*}
	\pm 0.5 - \frac{-0.25}{\sqrt{x_1 - x_1^2}} & = -0.25                                                                                 \\
	\frac{-0.25}{\sqrt{x_1 - x_1^2}}           & = \{0.75, -0.25\}                                                                       \\
	-\frac{1}{\sqrt{x_1 - x_1^2}}              & = \{3, -1\}                                                                             \\
	\left\{\frac{1}{9}, 1\right\}              & = x_1 - x_1^2                                                                           \\
	\text{Using } \frac{1}{9}: x_1             & = \left\{0.127322003750035, 0.872677996249965\right\}                                   \\
	\text{Using } 1: x_1                       & = \left\{\frac{1}{2} - \frac{\sqrt{3} i}{2}, \frac{1}{2} + \frac{\sqrt{3} i}{2}\right\}
\end{align*}
The largest real value between \((0, 1)\) for \(x_1 = 0.872678\)
\newpage
\subsection*{13D}
\begin{align*}
	P_3(x)                                                   & = \frac{1.216316 x \left(x - 1\right) \left(x - 0.5\right)}{0.25 \left(0.25 - 1\right) \left(0.25 - 0.5\right)}
	+ \frac{1.357008 x \left(x - 1\right) \left(x - 0.25\right)}{0.5 \left(0.5 - 1\right) \left(0.5 - 0.25\right)}                                                                                       \\
	                                                         & + \frac{1.381773 x \left(x - 0.5\right) \left(x - 0.25\right)}{1 \left(1 - 0.5\right) \left(1 - 0.25\right)}
	+ \frac{1.0 \left(x - 1\right) \left(x - 0.5\right) \left(x - 0.25\right)}{\left(- 1\right) \left(- 0.5\right) \left(0.25\right)}                                                                    \\
	                                                         & = 25.948083 x \left(x - 1\right) \left(x - 0.5\right)
	- 21.712130 x \left(x - 1\right) \left(x - 0.25\right)                                                                                                                                               \\
	                                                         & + 3.684729 x \left(x - 0.5\right) \left(x - 0.25\right) - 8.0 \left(x - 1\right) \left(x - 0.5\right) \left(x - 0.25\right)               \\
	R_3(x)                                                   & = \frac{f^4(\xi)}{4!}\left(x - 1\right) \left(x - 0.5\right) \left(x - 0.25\right)\left(x\right)                                          \\
	                                                         & = \frac{\sin{\left(\xi \right)} + \cos{\left(\xi \right)}}{24}\left(x - 1\right) \left(x - 0.5\right) \left(x - 0.25\right)\left(x\right) \\
	\left(\xi = \frac{\pi}{4}, x = 0.8316\right) \rightarrow & = 0.001591
\end{align*}
\subsection*{17}
\subsection*{19}
\subsubsection*{Interpolating polynomial}
\[P_5(x) = -\frac{10089 x^{5}}{4000000} + \frac{6001123 x^{4}}{240000} - \frac{2379665339 x^{3}}{24000} + \frac{471801682097 x^{2}}{2400} - \frac{116923918291129 x}{600} + 77269170756852\]
\subsubsection*{Estimates}
\begin{enumerate}
	\item 1950: 192539
	\item 1975: 215525
	\item 2014: 306214
	\item 2020: 266165
\end{enumerate}
The estimated 1950 from the interpolating polynomial was off by more than 40\%, while the 2014 figure was off by approximately 3\%. I would be skeptical about the estimates that I found in the interpolating polynomial.

\section*{Exercise Set 3.2}
\subsection*{1B}
\begin{align*}
	f(x) = - 1.333333 x        & \left(- 2.0 \left(- x - 0.75\right) \left(1.43875 x + 0.694625\right) - 2.0 \left(0.18825 x + 0.069375\right) \left(x + 0.25\right)\right)            \\
	                           & - 1.333333 \left(- x - 0.75\right) \left(- 2.0 x \left(1.43875 x + 0.694625\right) - 2.0 \left(- x - 0.5\right) \left(3.06425 x + 1.101\right)\right) \\
	f\left(-\frac{1}{3}\right) & = 0.174519
\end{align*}

\subsection*{3}
\subsubsection*{A}
\begin{align*}
	f(x)                      & = 3^x                            \\
	(x_0, x_1, x_2, x_3, x_4) & = (-2, -1, 0, 1, 2)              \\
	P(\sqrt{3})               & = 6.780246                       \\
	f(\sqrt{3})               & = 6.704992                       \\
	\text{Error (Part C)}     & = 6.780246 - 6.704992 = 0.075254
\end{align*}
\subsubsection*{B}
\begin{align*}
	f(x)                      & = \sqrt{x}                       \\
	(x_0, x_1, x_2, x_3, x_4) & = (0, 1, 2, 4, 5)                \\
	P(\sqrt{3})               & = 1.330337                       \\
	f(\sqrt{3})               & = 1.316074                       \\
	\text{Error (Part C)}     & = 1.330337 - 1.316074 = 0.014263
\end{align*}
\subsection*{5}
\begin{align*}
	P_2       & = 4.0  \\
	P_{1,2}   & = 3.2  \\
	P_{0,1,2} & = 3.08
\end{align*}
\subsection*{12}
\begin{align*}
	x            & = [0.3, 0.4, 0.5, 0.6]                                            \\
	y = x-e^{-x} & = \left[ -0.440818, \  -0.27032, \  -0.106531, \  0.051188\right] \\
	f^{-1}(0)    & = 0.567143
\end{align*}

\section*{3.3}
\subsection*{1B}
\begin{align*}
	P_1      & = -0.1769446 + 1.9069687(x-0.6)                                                             \\
	P_1(0.9) & = 0.395146                                                                                  \\
	P_2      & = -0.1769446 + 1.9069687(x-0.6) + 0.959224(x-0.7)(x-0.6)                                    \\
	P_2(0.9) & = 0.452700                                                                                  \\
	P_3      & = -0.1769446 + 1.9069687(x-0.6) + 0.959224(x-0.7)(x-0.6) + (-1.785741)(x-0.8)(x-0.7)(x-0.6) \\
	P_3(0.9) & = 0.441985
\end{align*}

\subsection*{3B}
Using \(x_1, x_2\) for \(P_1\), and \(x_1, x_2, x_3\) for \(P_2\)
\begin{align*}
	P_1       & = 2.905876x-0.574574                                                      \\
	P_1(0.25) & = 0.151895                                                                \\
	P_2       & = 2.905876x-2.438209(x-0.2)(x-0.1)-0.574574                               \\
	P_2(0.25) & = 0.133608                                                                \\
	P_3       & = 3.365129x-0.473152(x-0.3)(x-0.2)(x-0.1)-2.296264(x-0.2)(x-0.1)-0.957012 \\
	P_3(0.25) & = -0.132775
\end{align*}

\subsection*{5B}
Using \(x_1, x_2\) for \(P_1\), and \(x_1, x_2, x_3\) for \(P_2\)
\begin{align*}
	P_1       & = 2.905876 x - 0.574574                                                                                                                                \\
	P_1(0.25) & = 0.151895                                                                                                                                             \\
	P_2       & = 2.905876 x - 2.296264 \left(x - 0.3\right) \left(x - 0.2\right) - 0.865162                                                                           \\
	P_2(0.25) & = -0.132952                                                                                                                                            \\
	P_3       & = 2.418235 x - 0.473152 \left(x - 0.4\right) \left(x - 0.3\right) \left(x - 0.2\right) - 2.438209 \left(x - 0.4\right) \left(x - 0.3\right) - 0.718869 \\
	P_3(0.25) & = -0.132775                                                                                                                                            \\
\end{align*}


\subsection*{8}
\subsubsection*{A}
\[P_4(x) = 0.063016 x \left(x - 0.6\right) \left(x - 0.3\right) \left(x - 0.1\right) + 0.215 x \left(x - 0.3\right) \left(x - 0.1\right) + 0.5725 x \left(x - 0.1\right) + 1.0517 x - 6.0\]
\subsubsection*{B}
\begin{align*}
	P_5(x) = 0.014159 & x \left(x - 1\right) \left(x - 0.6\right) \left(x - 0.3\right) \left(x - 0.1\right) + 0.063016 x \left(x - 0.6\right) \left(x - 0.3\right) \\
	                  & \left(x - 0.1\right) + 0.215 x \left(x - 0.3\right) \left(x - 0.1\right) + 0.5725 x \left(x - 0.1\right) + 1.0517 x - 6.0
\end{align*}

\subsection*{16}
\begin{align*}
	f[x_0, x_1] & = 5.0 \\
	f[x_0]      & = 1.0 \\
	f[x_1]      & = 3.0
\end{align*}

\subsection*{18}
\subsubsection*{A}
\[P_4(0.75) = 72.86 \rightarrow \text{1 minute, 12.86 seconds}\]
The actual time was 1:13, so this estimate is extremely close.
\subsubsection*{B}
\[
	\frac{d}{dx}P_4(1.25) = 89.72\frac{\text{seconds}}{\text{mile}} \approx
	40.12 \text{ miles per hour}
\]

\subsection*{20}
\begin{tabular}{lrllr}
	\toprule
	{} & x    & P(x) & Q(x) & Actual \\
	\midrule
	0  & -2.0 & -1   & -1   & -1     \\
	1  & -1.0 & 3    & 3    & 3      \\
	2  & 0.0  & 1    & 1    & 1      \\
	3  & 1.0  & -1   & -1   & -1     \\
	4  & 2.0  & 3    & 3    & 3      \\
	\bottomrule
\end{tabular} \\
\(P(x)\) and \(Q(x)\) are the same function - when you reduce the two equations they are equivalent. Thus, \(P(x)\) does not violate the uniqueness property of interpolating polynomials.
\subsection*{21}
\begin{align*}
	f[x_{2}]                                                                                                                        & = a_{2} \left(- x_{0} + x_{2}\right) \left(- x_{1} + x_{2}\right) + f[x_{0}] + f[x_{1}] \left(- x_{0} + x_{2}\right) \\
	- f[x_{0}] - f[x_{1}] \left(- x_{0} + x_{2}\right) + f[x_{2}]                                                                   & = a_{2} \left(- x_{0} + x_{2}\right) \left(- x_{1} + x_{2}\right)                                                    \\
	\frac{- f[x_{0}] - f[x_{1}] \left(- x_{0} + x_{2}\right) + f[x_{2}]}{\left(- x_{0} + x_{2}\right) \left(- x_{1} + x_{2}\right)} & = a_{2}                                                                                                              \\
	- \frac{f[x_{1}]}{- x_{1} + x_{2}} + \frac{- f[x_{0}] + f[x_{2}]}{\left(- x_{0} + x_{2}\right) \left(- x_{1} + x_{2}\right)}    & = a_{2}                                                                                                              \\
	f[x_{0},x_{1},x_{2}]                                                                                                            & = a_{2}
\end{align*}
\subsection*{22}
\begin{align*}
	f(x) = P_{n+1}(x)                                            & = P_n(x) + f[x_0, x_1,...,x_n, x](x-x_0)(x-x_1)...(x-x_n) \\
	\text{Substituting \(f(x)\)}:                                &                                                           \\
	P_n(x) + \frac{f^{n+1}(\xi)}{(n+1)!}(x-x_0)(x-x_1)...(x-x_n) & = P_n(x) + f[x_0, x_1,...,x_n, x](x-x_0)(x-x_1)...(x-x_n) \\
	\frac{f^{n+1}(\xi)}{(n+1)!}                                  & = f[x_0, x_1,...,x_n, x]                                  \\
\end{align*}

\section*{Exercise Set 8.3}
\subsection*{1A / 3A}
\begin{align*}
	(x_0, x_1, x_2) & = \left(-\frac{3}{\sqrt{2}}, 0,\frac{3}{\sqrt{2}}\right) \\
	P_3(x)          & = 0.532042 x \left(x + 0.866025\right) + 0.66901 x + 1.0 \\
	f(x) - P(x)     & = \frac{e^\xi}{3!} * \frac{1}{2^2} = 0.1132617           \\
\end{align*}

\section*{Exercise Set 3.4}
\subsection*{1C}
\begin{align*}
	P_5(x) = -0.024751 & +0.751(x + 0.5)+2.751\left(x + 0.5\right)^{2}+\left(x + 0.25\right) \left(x + 0.5\right)^{2}                                                 \\
	                   & -7.105427 * 10^{-15}\left(x + 0.25\right)^{2} \left(x + 0.5\right)^{2}+2.131628*10^{-14}x \left(x + 0.25\right)^{2} \left(x + 0.5\right)^{2}
\end{align*}
\subsection*{5}
\subsubsection*{A / B}
\begin{align*}
	H_2(x)     & = 0.29552+0.95534(x - 0.3)+-0.142\left(x - 0.3\right)^{2}+-1.05\left(x - 0.32\right) \left(x - 0.3\right)^{2}                                  \\
	           & +20.77778\left(x - 0.32\right)^{2} \left(x - 0.3\right)^{2}+-436.29630\left(x - 0.35\right) \left(x - 0.32\right)^{2} \left(x - 0.3\right)^{2} \\
	H_2(0.34)  & = 0.33349                                                                                                                                      \\
	\sin{0.34} & = 0.33349                                                                                                                                      \\
	R_2(0.34)  & = 0.000003
\end{align*}
With 5 digit rounding, the error bound exceeds the actual error of 0.

\subsubsection*{C}
\begin{align*}
	H_7(x)     & = 0.29552+0.95534(x - 0.3)+-0.142\left(x - 0.3\right)^{2}+-1.05\left(x - 0.32\right) \left(x - 0.3\right)^{2}                                                                                              \\
	           & -32.77778\left(x - 0.32\right)^{2} \left(x - 0.3\right)^{2}+17574.07407\left(x - 0.33\right) \left(x - 0.32\right)^{2} \left(x - 0.3\right)^{2}                                                            \\
	           & -744814.81482\left(x - 0.33\right)^{2} \left(x - 0.32\right)^{2} \left(x - 0.3\right)^{2}+29455555.55570\left(x - 0.35\right) \left(x - 0.33\right)^{2} \left(x - 0.32\right)^{2} \left(x - 0.3\right)^{2} \\
	H_7(0.34)  & = 0.33349                                                                                                                                                                                                  \\
	\sin{0.34} & = 0.33349                                                                                                                                                                                                  \\
	R_7(0.34)  & = 3.75264 * 10^{-19}
\end{align*}

\subsection*{9}
\begin{align*}
	H_9(x)   & = 75x+-0.031111x^{2} \left(x - 3\right)^{2}+-0.006444x^{2} \left(x - 5\right) \left(x - 3\right)^{2}+0.002264x^{2} \left(x - 5\right)^{2} \left(x - 3\right)^{2}   \\
	         & -0.000913x^{2} \left(x - 8\right) \left(x - 5\right)^{2} \left(x - 3\right)^{2}+0.000131x^{2} \left(x - 8\right)^{2} \left(x - 5\right)^{2} \left(x - 3\right)^{2} \\
	         & -2.022363*10^{-5}x^{2} \left(x - 13\right) \left(x - 8\right)^{2} \left(x - 5\right)^{2} \left(x - 3\right)^{2}                                                    \\
	H_9(10)  & = 742.50                                                                                                                                                           \\
	H_9'(10) & = 48.38                                                                                                                                                            \\
\end{align*}
The car surpasses 55 miles per hour at approximately \(5.65147\) seconds.
\subsection*{10}

The given divided difference table provides the coefficients to write the Hermite polynomial using the Newton's Divided Difference form of the interpolating polynomial.
\begin{align*}
	a_0 & = f[z_0] = f(x_0)                                                                                   \\
	a_1 & = f[z_0, z_1] = f'(x_0)                                                                             \\
	a_2 & = f[z_0, z_1, z_2] = \frac{f[z_1, z_2] (x - z_0) - f[z_0, z_1] (x - z_2)}{z_2 - z_1}                \\
	a_3 & = f[z_0, z_1, z_2, z_3] = \frac{f[z_1, z_2, z_3] (x - z_0) - f[z_0, z_1, z_2] (x - z_3)}{z_3 - z_0} \\
\end{align*}

Using these coefficients:
\begin{align*}
	H_3(x) & = f[z_0] + f[z_0, z_1] (x-x_0) + f[z_0, z_1, z_2] (x-x_0)^2 + f[z_0, z_1, z_2, z_3](x-x_0)^2(x-x_1) \\
\end{align*}

\section*{Exercise Set 3.5}
\subsection*{3C}
\begin{tabular}{l|cccc}
	i & a           & b          & c           & d           \\
	\toprule{}
	0 & -0.02475000 & 1.03237500 & -0.00000000 & 6.50200000  \\
	1 & 0.33493750  & 2.25150000 & 4.87650000  & -6.50200000
\end{tabular}
\subsection*{11}
\begin{align*}
	3 x^{2} - 2 & = - b - 2 c \left(x - 1\right) - 3 d \left(x - 1\right)^{2} \\
	6 x         & = - 2 c - 6 d \left(x - 1\right)                            \\
	d           & = -1                                                        \\
	c           & = -3                                                        \\
	b           & = -1
\end{align*}
\subsection*{15}
\begin{align*}
	S_0                                        & = 2.103418 x + 1.0      \\
	S_1                                        & = 2.324637 x + 0.988939 \\
	\int_{0}^{0.05}S_0 + \int_{0.05}^{0.1} S_1 & = 0.110794              \\
	\int_{0}^{0.1} e^x                         & = 0.110701              \\
	|Error|                                    & = 9.223578 * 10^{-5}
\end{align*}
\subsection*{17}

\begin{tabular}{l|cccc}
	Range      & a     & b     & c     & d     \\
	\toprule{}
	(0,0.25)   & 1.00  & -0.76 & -0.00 & -6.63 \\
	(0.25,0.5) & 0.71  & -2.00 & -4.97 & 6.63  \\
	(0.5,0.75) & 0.00  & -3.24 & -0.00 & 6.63  \\
	(0.75,1)   & -0.71 & -2.00 & 4.97  & -6.63
\end{tabular}

\begin{align*}
	\int_{0}^{1}S(x)  & = -0.493792      \\
	\int_{0}^{1}f(x)  & = 0              \\
	|Error|           & = 0.493792       \\[3.5ex]
	S'(0.5) = f'(0.5) & = 0              \\
	|Error|           & = 0              \\[3.5ex]
	S''(0.5)          & = 0              \\
	f''(0.5)          & = -\frac{\pi}{2} \\
	|Error|           & = -\frac{\pi}{2} \\
\end{align*}

\section*{Exercise Set 3.6}
\subsection*{3A}
\begin{tabular}{l|cccc}
	  & 0    & 1      & 2     & 3      \\
	\toprule{}
	a & 1.00 & 1.50   & 15.00 & -11.50 \\
	b & 6.00 & -14.25 & 19.50 & -9.25
\end{tabular}

\section*{Exercise Set 4.1}
\subsection*{1}
\subsubsection*{A}
\(
\begin{bmatrix}
	0.85 & 0.80
\end{bmatrix}
\)
\subsubsection*{B}
\(
\begin{bmatrix}
	3.71 & 3.15
\end{bmatrix}
\)

\subsection*{3}
\subsubsection*{A}
\(
\begin{bmatrix}
	x        & |Error|  & Bound    \\
	0.000000 & 0.294000 & 0.300000 \\
	0.200000 & 0.284597 & 0.277860 \\
	0.400000 & 0.259175 & 0.250818
\end{bmatrix}
\)

\subsubsection*{B}
\(
\begin{bmatrix}
	x        & |Error|  & Bound    \\
	0.000000 & 0.294000 & 0.300000 \\
	0.200000 & 0.284597 & 0.277860 \\
	0.400000 & 0.259175 & 0.250818
\end{bmatrix}
\)

\subsection*{13}
Using the five point midpoint formula:
\begin{align*}
	f'(x)                                 & = \frac{1}{12}\left(f(x - 2h) - 8(f(x-h)) + 8 (f(x+h)) - f(x+h)\right) \\
	f'(3)                                 & = 0.22585                                                              \\
	|Error| = \frac{f^5(\xi) * (1)^4}{30} & = 0.766667                                                             \\
\end{align*}

\subsection*{15}
\subsubsection*{A}
\begin{align*}
	\frac{f(x_1) - f(x_0)}{x_1 - x_0} & = 0.852 \\
	\frac{f(x_2) - f(x_1)}{x_2 - x_1} & = 0.796 \\[3.5ex]
	\frac{f(x_0) - f(x_1)}{x_0 - x_1} & = 0.852 \\
	\frac{f(x_1) - f(x_2)}{x_1 - x_2} & = 0.796 \\
\end{align*}

\subsubsection*{B}
\begin{align*}
	\frac{f(x_1) - f(x_0)}{x_1 - x_0} & = 3.707 \\
	\frac{f(x_2) - f(x_1)}{x_2 - x_1} & = 3.153 \\[3.5ex]
	\frac{f(x_0) - f(x_1)}{x_0 - x_1} & = 3.707 \\
	\frac{f(x_1) - f(x_2)}{x_1 - x_2} & = 3.153 \\
\end{align*}

\subsection*{18A}
\begin{align*}
	f(x)     & = - 0.339961 x^{4} + 0.428736 x^{3} - 0.770008 x^{2} + 0.072283 x + 0.993323 \\
	f'(x)    & = - 1.359844 x^{3} + 1.286209 x^{2} - 1.540016 x + 0.072283                  \\
	f'(0.4)  & = -0.424960                                                                  \\
	f''(x)   & = - 4.079531 x^{2} + 2.572419 x - 1.540016                                   \\
	f''(0.4) & = -1.163773                                                                  \\
\end{align*}

\subsection*{24}
\begin{align*}
	f'(x_0)         & = A\left(f(x_0) + (-h)f'(x_0) + \frac{f''(x_0)(-h)^2}{2!} + \frac{f'''(x_0)(-h)^3}{3!} + \frac{f^4(x_0)(-h)^4}{4!} + \frac{f^5(\xi)(-h)^5}{5!}\right)   \\
	                & + B\left(f(x_0)\right)                                                                                                                                  \\
	                & + C\left(f(x_0) + ((h))f'(x_0) + \frac{f''(x_0)(h)^2}{2!} + \frac{f'''(x_0)(h)^3}{3!} + \frac{f^4(x_0)(h)^4}{4!} + \frac{f^5(\xi)(h)^5}{5!}\right)      \\
	                & + D\left(f(x_0) + ((2h))f'(x_0) + \frac{f''(x_0)(2h)^2}{2!} + \frac{f'''(x_0)(2h)^3}{3!} + \frac{f^4(x_0)(2h)^4}{4!} + \frac{f^5(\xi)(2h)^5}{5!}\right) \\
	                & + E\left(f(x_0) + ((3h))f'(x_0) + \frac{f''(x_0)(3h)^2}{2!} + \frac{f'''(x_0)(3h)^3}{3!} + \frac{f^4(x_0)(3h)^4}{4!} + \frac{f^5(\xi)(3h)^5}{5!}\right) \\
	                & = (A+B+C+D+E) f(x_0)                                                                                                                                    \\
	                & + (-hA + hC + 2hD + 3hE) (f'(x_0))                                                                                                                      \\
	                & + (0.5(h^2)A + 0.5(h^2)C + 2h^2D + 4.5h^2E)                                                                                                             \\
	                & + \frac{1}{6} (-h^3A + h^3C + 8h^3D + 27h^3)                                                                                                            \\
	                & + \frac{1}{24} (h^4A + h^4C + 16h^4D + 81h^4E)                                                                                                          \\
	                & \text{Using } h=1:                                                                                                                                      \\
	(A, B, C, D, E) & = (-0.25, -0.83333333,  1.5,        -0.5,         0.08333333)                                                                                           \\
\end{align*}
The polynomial:
\[
	f'(x_0) = - \frac{0.25}{h}(h + x_{0})- \frac{0.83333333}{h}(x_{0}) + \frac{1.5}{h}(- h + x_{0})- \frac{0.5}{h}(- 2 h + x_{0}) + \frac{0.08333333}{h}(- 3 h + x_{0}) + O(h^4)
\]
\subsection*{25}
\begin{align*}
	f'(0.4) & = - 1.25 f{\left(0.2 \right)} - 4.16666665 f{\left(0.4 \right)} + 7.5 f{\left(0.6 \right)} - 2.5 f{\left(0.8 \right)} + 0.41666665 f{\left(1.0 \right)} \\
	        & = -0.424984                                                                                                                                             \\
	f'(0.8) & = - 1.25 f{\left(0.6 \right)} - 4.16666665 f{\left(0.8 \right)} + 7.5 f{\left(1.0 \right)} - 2.5 f{\left(1.2 \right)} + 0.41666665 f{\left(1.4 \right)} \\
	        & = -1.032772
\end{align*}
\subsection*{29}
\begin{align*}
	e(h)                & = \frac{\epsilon}{h} + \frac{h^2}{6}M    \\
	e'(h)               & = \frac{M h}{3} - \frac{\epsilon}{h^{2}} \\
	\text{Using } e'(h) & = 0                                      \\
	h                   & = \sqrt[3]{\frac{3\epsilon}{M}}          \\
\end{align*}
If \(e'(h) < 0, h < \sqrt[3]{\frac{3\epsilon}{M}}\), and if \(e'(h) > 0, h > \sqrt[3]{\frac{3\epsilon}{M}}\). Thus, there is a minimum at \( h= \sqrt[3]{\frac{3\epsilon}{M}} \).

\newpage
\subsection*{27}
Using \(x = 420\) and \(f(x) = x^4\): \\
\begin{tabular}{l|rrr}
	n  & \(f_n'(x)\)               & \(f(x)\)  & Absolute Error          \\
	\toprule
	0  & 297412081.000000000000000 & 296352000 & 1060081.000000000000000 \\
	1  & 296457856.800999999046326 & 296352000 & 105856.800999999046326  \\
	2  & 296362584.168000996112823 & 296352000 & 10584.168000996112823   \\
	3  & 296353058.401679992675781 & 296352000 & 1058.401679992675781    \\
	4  & 296352105.840016782283783 & 296352000 & 105.840016782283783     \\
	5  & 296352010.584000170230865 & 296352000 & 10.584000170230865      \\
	6  & 296352001.058399975299835 & 296352000 & 1.058399975299835       \\
	7  & 296352000.105839967727661 & 296352000 & 0.105839967727661       \\
	8  & 296352000.010583996772766 & 296352000 & 0.010583996772766       \\
	9  & 296352000.001058399677277 & 296352000 & 0.001058399677277       \\
	10 & 296352000.000105857849121 & 296352000 & 0.000105857849121       \\
	11 & 296352000.000010609626770 & 296352000 & 0.000010609626770       \\
	12 & 296352000.000001072883606 & 296352000 & 0.000001072883606       \\
	13 & 296352000.000000119209290 & 296352000 & 0.000000119209290       \\
	14 & 296352000                 & 296352000 & 0                       \\
	15 & 296352000                 & 296352000 & 0                       \\
	16 & 296352000                 & 296352000 & 0                       \\
	17 & 296352000                 & 296352000 & 0                       \\
	18 & 296352000                 & 296352000 & 0                       \\
	19 & 296352000                 & 296352000 & 0                       \\
	20 & 296352000                 & 296352000 & 0                       \\
\end{tabular}

As \(n\) increases, \(f_n'(x)\) comes closer to the actual value of \(f(x)\).

\section*{4.2}
\subsection*{1}
\subsubsection*{A}

Output from program:
\[
	\left[
		\begin{array}{l|ccc}
			h           & N_1      & N_2      & N_3      \\
			\toprule{}
			h           & 0.841181 & 0.000000 & 0.000000 \\
			\frac{h}{2} & 0.911608 & 0.982035 & 0.000000 \\
			\frac{h}{4} & 0.953102 & 0.994596 & 1.007157
		\end{array}
		\right]
\]
\[N_3(h) = 1.007157\]
\newpage
\subsubsection*{C}
\[\left[
		\begin{array}{l|ccc}
			h           & N_1      & N_2      & N_3      \\
			\toprule{}
			h           & 2.290365 & 0.000000 & 0.000000 \\
			\frac{h}{2} & 2.305264 & 2.305501 & 0.000000 \\
			\frac{h}{4} & 2.295243 & 2.295084 & 2.295074
		\end{array}
		\right]\]
\[N_3(h) = 2.250282\]

\subsection*{5}
\[
	\left[
		\begin{array}{l|rrrr}
			h           & N_1      & N_2      & N_3      & N_4      \\
			\toprule{}
			h           & 1.570796 & 0.000000 & 0.000000 & 0.000000 \\
			\frac{h}{2} & 1.896119 & 1.901283 & 0.000000 & 0.000000 \\
			\frac{h}{4} & 1.974232 & 1.975472 & 1.975544 & 0.000000 \\
			\frac{h}{8} & 1.993570 & 1.993877 & 1.993895 & 1.993896
		\end{array}
		\right]
\]
\[
	N_4(h) = 1.993896
\]

\subsection*{8}
To remove fluff from the calculations, let's define some variables:
\begin{align*}
	A                                                                                                                             & = \frac{1}{2}f''(x_0)                                                                                                    \\
	B                                                                                                                             & = \frac{1}{6}f^3(x_0)                                                                                                    \\
	D(a)                                                                                                                          & = f(x_0 + a) - f(x_0)                                                                                                    \\[3ex]
	\operatorname{N_{1}}{\left(h \right)}                                                                                         & = - A h - B h^{2} + \mathcal{O}{\left(h^{3} \right)} + \frac{D{\left(h \right)}}{h}                                      \\
	\operatorname{N_{1}}{\left(2 h \right)}                                                                                       & =- 2 A h - 4 B h^{2} + \mathcal{O}{\left(h^{3} \right)} + \frac{D{\left(2 h \right)}}{2 h}                               \\
	\operatorname{N_{2}}{\left(h \right)} = 2 * \operatorname{N_{1}}{\left( h \right)} - \operatorname{N_{1}}{\left(2 h \right)}  & = 2 B h^{2} + \mathcal{O}{\left(h^{3} \right)} + \frac{2 D{\left(h \right)}}{h} - \frac{D{\left(2 h \right)}}{2 h}       \\
	\operatorname{N_{2}}{\left(2 h \right)}                                                                                       & =8 B h^{2} + \mathcal{O}{\left(h^{3} \right)} + \frac{D{\left(2 h \right)}}{h} - \frac{D{\left(4 h \right)}}{4 h}        \\
	\operatorname{N_{3}}{\left(h \right)}  = 4 * \operatorname{N_{2}}{\left( h \right)} - \operatorname{N_{2}}{\left(2 h \right)} & = \mathcal{O}{\left(h^{3} \right)} + \frac{32 D{\left(h \right)} - 12 D{\left(2 h \right)} + D{\left(4 h \right)}}{12 h} \\
\end{align*}
Substituting \(D(a)\) back into the equation:
\[
	\operatorname{N_{3}}{\left(h \right)}  =\mathcal{O}{\left(h^{3} \right)} + \frac{- 21 f{\left(x \right)} + 32 f{\left(h + x \right)} - 12 f{\left(2 h + x \right)} + f{\left(4 h + x \right)}}{12 h} + \mathcal{O}{\left(h^{3} \right)}
\]


\subsection*{9}
\begin{align*}
	N_1(h)   & = K_{1} h + K_{2} h^{2} + D{\left(h \right)}                                                                            \\
	N_1(h/3) & = \frac{K_{1} h}{3} + \frac{K_{2} h^{2}}{9} + D{\left(\frac{h}{3} \right)}                                              \\
	N_1(h/9) & = \frac{K_{1} h}{9} + \frac{K_{2} h^{2}}{81} + D{\left(\frac{h}{9} \right)}                                             \\
	N_2(h)   & = - \frac{K_{2} h^{2}}{3} + \frac{3 D{\left(\frac{h}{3} \right)}}{2} - \frac{D{\left(h \right)}}{2}                     \\
	N_2(h/3) & = - \frac{K_{2} h^{2}}{27} + \frac{3 D{\left(\frac{h}{9} \right)}}{2} - \frac{D{\left(\frac{h}{3} \right)}}{2}          \\
	N_3(h)   & = \frac{27 D{\left(\frac{h}{9} \right)}}{16} - \frac{3 D{\left(\frac{h}{3} \right)}}{4} + \frac{D{\left(h \right)}}{16} \\
\end{align*}

\subsection*{11}
\subsubsection*{A / B / C}
\begin{align*}
	N_1(h)    & = K_{1} h + K_{2} h^{2} + K_{3} h^{3} + \left(h + 1\right)^{\frac{1}{h}}                                                                                                     \\
	N_1(0.04) & = 0.04 K_{1} + 0.0016 K_{2} + 6.4 \cdot 10^{-5} K_{3} + 2.665836                                                                                                             \\
	N_2(h)    & = - \frac{K_{2} h^{2}}{2} - \frac{3 K_{3} h^{3}}{4} + 2 \left(\frac{h}{2} + 1\right)^{\frac{2}{h}} - \left(h + 1\right)^{\frac{1}{h}}                                        \\
	N_2(0.04) & = - 0.0008 K_{2} - 4.8 \cdot 10^{-5} K_{3} + 2.71734                                                                                                                         \\
	N_3(h)    & = \frac{K_{3} h^{3}}{8} + \frac{8 \left(\frac{h}{4} + 1\right)^{\frac{4}{h}}}{3} - 2 \left(\frac{h}{2} + 1\right)^{\frac{2}{h}} + \frac{\left(h + 1\right)^{\frac{1}{h}}}{3} \\
	N_3(0.04) & = 8.0 \cdot 10^{-6} K_{3} + 2.718273                                                                                                                                         \\
	e         & = 2.718282                                                                                                                                                                   \\
\end{align*}
The assumption in B seems to hold up. The absolute error between \(N_3\) and the exact answer is approximately \(0.000009\).

\section*{Exercise Set 4.3}
\subsection*{1 / 3}
Using the trapezoidal rule:
\[
	\frac{h}{2}\left(f(a) + f(a+h)\right)
\]
\subsubsection*{A}
\[X = [0.5, 1], Y = [0.0625, 1] \rightarrow 0.265625\]
\[
	\text{Error Bound} =  0.125000000000000
\]
\[
	\text{Actual Error} =  0.071875
\]
\subsubsection*{C}
\[
	X = [1, 1.5], Y = [0.0, 0.912296]\rightarrow  0.228074
\]
\[
	\text{Error Bound} =  0.0396971897522534
\]
\[
	\text{Actual Error} =  0.03581476557804644
\]
\subsubsection*{H}
\[
	X = [0, 0.785398], Y = [0.0, 10.550724] \rightarrow 4.143260
\]
\[
	\text{Error Bound} =  2.12980904832081
\]
\[
	\text{Actual Error} =  1.5546310226869071
\]
\subsection*{19}
\(
\begin{array}{l|rrr}
	\toprule
	i & x^i & \int_{-1}^{1}f(x) & f{\left(- \frac{\sqrt{3}}{3} \right)} + f{\left(\frac{\sqrt{3}}{3} \right)} \\
	\midrule
	0 & 1   & 2                 & 2                                                                           \\
	1 & x   & 0                 & 0                                                                           \\
	2 & x^2 & 2/3               & 2/3                                                                         \\
	3 & x^3 & 0                 & 0                                                                           \\
	4 & x^4 & 2/5               & 2/9                                                                         \\
	\bottomrule
\end{array}
\)\\
At \(x^4\), the values diverge. Thus, the degree of precision is 3.
\subsection*{21}
\(
\begin{array}{l|rrr}
	\toprule
	i & x^i = f(x) & c_0f(-1) + c_1f(1) + c_2(1) & \int_{-1}^{1}f(x) \\
	\midrule
	0 & 1          & c_0 + c_1 + c_2             & 2                 \\
	1 & x          & -c_0 + c_2                  & 0                 \\
	2 & x^2        & c_0 + c_2                   & 2/3               \\
	\bottomrule
\end{array}
\) \\[5ex]
Solving for the unknowns:
\begin{align*}
	c_{0} + c_{1} + c_{2} - 2   & = 0                                                         \\
	- c_{0} + c_{2}             & = 0                                                         \\
	c_{0} + c_{2} - \frac{2}{3} & = 0                                                         \\
	(c_0, c_1, c_2)             & = \left( \frac{1}{3}, \  \frac{4}{3}, \  \frac{1}{3}\right) \\
\end{align*}
\subsection*{26}
\[
	\begin{array}{llll}
		\toprule
		n & x^n   & \int_{x_0}^{x_2}f(x)                                         & a_0f(x_0) + a_1f(x_1) +a_2f(x_2)                                                                \\
		\midrule
		0 & x     & - \frac{x_{0}^{2}}{2} + \frac{\left(h + x_{0}\right)^{2}}{2} & a_{0} x_{0} + a_{1} \left(\frac{h}{2} + x_{0}\right) + a_{2} \left(h + x_{0}\right)             \\
		1 & x^{2} & - \frac{x_{0}^{3}}{3} + \frac{\left(h + x_{0}\right)^{3}}{3} & a_{0} x_{0}^{2} + a_{1} \left(\frac{h}{2} + x_{0}\right)^{2} + a_{2} \left(h + x_{0}\right)^{2} \\
		2 & x^{3} & - \frac{x_{0}^{4}}{4} + \frac{\left(h + x_{0}\right)^{4}}{4} & a_{0} x_{0}^{3} + a_{1} \left(\frac{h}{2} + x_{0}\right)^{3} + a_{2} \left(h + x_{0}\right)^{3} \\
		\bottomrule
	\end{array}
\]
\begin{align*}
	- \frac{x_{0}^{2}}{2} + \frac{\left(h + x_{0}\right)^{2}}{2} & = a_{0} x_{0} + a_{1} \left(\frac{h}{2} + x_{0}\right) + a_{2} \left(h + x_{0}\right)             \\
	- \frac{x_{0}^{3}}{3} + \frac{\left(h + x_{0}\right)^{3}}{3} & = a_{0} x_{0}^{2} + a_{1} \left(\frac{h}{2} + x_{0}\right)^{2} + a_{2} \left(h + x_{0}\right)^{2} \\
	- \frac{x_{0}^{4}}{4} + \frac{\left(h + x_{0}\right)^{4}}{4} & = a_{0} x_{0}^{3} + a_{1} \left(\frac{h}{2} + x_{0}\right)^{3} + a_{2} \left(h + x_{0}\right)^{3} \\
	(a_0, a_1, a_2)                                              & = \left( \frac{h}{6}, \  \frac{2 h}{3}, \  \frac{h}{6}\right)
\end{align*}
\end{document}